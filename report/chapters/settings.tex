\subsection{Training Programs}
The training programs are defined as follows and they are the typical training sessions that a runner would do during his weekly training plan:
\begin{itemize}
  \item \textbf{Fartlek} – Represent a variable-intensity workout where the athlete alternates between high and low intensity segments, typically in a short time alternation pattern.
  \item \textbf{Progression} – A typical workout where the athlete gradually increases the pace over a set distance or time, starting at a comfortable speed and finishing at a faster pace.
  \item \textbf{Endurance} – A long, steady-state run at a moderate pace, designed to build aerobic capacity and endurance.
  \item \textbf{Recovery} – A low-intensity workout aimed at promoting recovery after a hard training session, typically involving easy running or walking, but with a focus on maintaining the athelte moving with a low heart rate and minimizing fatigue.
\end{itemize}
All these workouts are defined in the \texttt{trainings.json} file.

\subsection{Athlete Archetypes}
The athlete archetypes are defined by their physiological parameters like the heart rate value at rest, the maximal heart rate able to reach and the weight. There are also two \texttt{fitness-related} values: 
\begin{itemize}
\item \textbf{FTP} – Functional Threshold Power (FTP) is a key metric used to define an athlete's performance profile. It represents the highest average power an athlete can sustain for about 60 minutes and is crucial for estimating personalized training zones and overall aerobic capacity. While FTP is most commonly associated with cycling, it is also relevant in running and other endurance sports. Because not every athlete knows their FTP, many fitness apps and smartwatches can estimate this value automatically by analyzing data from multiple workouts, regardless of the activity type. This makes FTP a practical and widely accessible measure for assessing and tracking athletic performance.
\item \textbf{Fitness factor} – A value that represents the athlete's fitness level, which is into the range of 0,7 for an elite athlete, up to 1,3 for an amatour athlete. This value is used to determine the athlete's ability to sustain high-intensity efforts, manage fatigue and how well the athlete it's able to recover.
\end{itemize}

The archetypes, defined in the \texttt{athletes.json} file, are the following:
\begin{itemize}
  \item \textbf{Elite} – Represents a highly trained athlete, like olympic runners with a low resting heart rate, high maximum heart rate, and high FTP. This archetype is characterized by its ability to sustain high-intensity efforts and recover quickly.
  \item \textbf{Runner} – Represents an athlete with average training, having typical resting and maximum heart rates, and a moderate FTP. This archetype can handle standard training sessions and recovers at a normal rate. Could be someone who has been training for a while, like several months, but is not at the elite level yet.
  \item \textbf{Amateur} – Represents an untrained or recreational athlete with a high resting heart rate, low maximum heart rate, and low FTP, typical of someone who begins to train. This archetype is characterized by its limited ability to sustain high-intensity efforts and recover slowly, is the one who suffers more advanced programs.
\end{itemize}

\subsection{GPX Track}\label{sec:gpx-track}
Training was performed on a real GPX track of the \textit{Parco degli Acquedotti} in Rome. To probe generalisation, the learned policies were replayed unchanged on two unseen yet topographically comparable routes: a riverside path in \textit{Parco Belfiore} and the \textit{Lago di Mezzo} waterfront loop, both in Mantova. These additional circuits share similar average slope with the first one but differ in curve geometry and surface, allowing verification that the agent’s behaviour is track-agnostic rather than over-fitted to the training venue. The results of all these simulation can be seen in the video folder.

